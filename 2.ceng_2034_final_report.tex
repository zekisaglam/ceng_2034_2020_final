\documentclass[onecolumn]{article}
%\usepackage{url}
%\usepackage{algorithmic}
\usepackage[a4paper]{geometry}
\usepackage{datetime}
\usepackage[margin=2em, font=small,labelfont=it]{caption}
\usepackage{graphicx}
\usepackage{mathpazo} % use palatino
\usepackage[scaled]{helvet} % helvetica
\usepackage{microtype}
\usepackage{amsmath}
\usepackage{subfigure}
% Letterspacing macros
\newcommand{\spacecaps}[1]{\textls[200]{\MakeUppercase{#1}}}
\newcommand{\spacesc}[1]{\textls[50]{\textsc{\MakeLowercase{#1}}}}

\title{\spacecaps{Assignment Report 2: Multiprocess Implementation}\\ \normalsize \spacesc{CENG2034, Operating Systems} }

\author{Zeki Sağlam\\zzekisaglam@gmail.com}
%\date{\today\\\currenttime}
\date{\today}

\begin{document}
\maketitle

\begin{abstract}
The purpose of this assignment is to understand child and parent relationships and use multiprocessing.
method.I downloaded
some files with the function using child process from a given Array list.
\end{abstract}


\section{Introduction}
This Assessment I compared many types of processes.I learned how to use it. 

\section{Assignments}
I started this task by taking the necessary libraries.


\includegraphics[scale=0.40]{1.png}



\subsection*{2.1. For Print parent id }

Print parent id to screen.Create child process using fork.If pid of process is greater than 0 that means it is parent process.os.wait() method is used by a process to wait for completion of a child process.
If pid of process is equal to 0 that means it is child process.Print child id to screen.


\includegraphics[scale=0.40]{2.png}

\subsection*{2.2With the child process, downloading the files via the given Array list}
Requests.get method makes a request to a url, and return the status code. If status code equal to 200 that means url is working. 

 Create a file named fille name. 
 Print information and file name after that Print if url was not downloaded, and check for duplicates.The elements registered in the a list were checked with the multiprocess technique and the duplicate
  
   \includegraphics[scale=0.45]{3.png}
    
  \includegraphics[scale=0.45]{4.png}
  
 
 
 
 
\subsection*{2.3 Avoiding the orphan process situation}

This time the problem was avoiding the orphan process situation. I added wait() for this, after
the child but before the parent because I wanted my parent process to wait child to complete.We created child proccesses and with ”os.wait()” methods , avoid from orphan processes.



\includegraphics[scale=0.45]{6.png}


\section{Results}
I got this output showing parent and child pids.
It shows a list of downloaded files, the pids of the processes in the functions, duplicate files list and taken time.
\includegraphics[scale=0.55]{Results.png}




\section{Conclusion}
I learned a lot with this task. One of the first things I learned; in multiprocessing, multiple processes or jobs can be run and managed by the CPU or a single program. 
This Assignment was very instructive and made me understand how processes work.




\section{My github}
github.com/zekisaglam/


\nocite{*}
\bibliographystyle{plain}
\bibliography{references}
\end{document}